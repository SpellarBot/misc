\documentclass[12pt, a4paper]{article}%{{{
\usepackage{tipa}
\usepackage{amsfonts}
\usepackage{amsmath}
\usepackage{mathrsfs}
\usepackage{amssymb}
\usepackage{latexsym, lineno, caption2, indentfirst}
\usepackage[super,square,comma,numbers,sort&compress]{natbib}
\usepackage{hyperref}
\usepackage{graphicx,color,overpic,subfigure}
\usepackage[rflt]{floatflt}
\usepackage{multirow}
\usepackage{endfloat} %[nomarkers]
\usepackage{diagbox}
%%% ------------------------------
\setlength{\topmargin}{-1cm} \setlength{\oddsidemargin}{0mm}
\textwidth 16cm \textheight 24cm \parskip=6pt
\newcommand{\newsection}[1]{\section {#1} \setcounter{equation}{0}}
\renewcommand{\thefootnote}{\fnsymbol{footnote}}
\renewcommand{\baselinestretch}{1.5}
\renewcommand{\textfraction}{0.3}
%%% ------------------------------
\citestyle{nature}
\begin{document}
\newcommand{\lr}[1]{\langle #1 \rangle}
\newcommand{\llr}[1]{\langle \hspace{-2.5pt} \langle #1 \rangle \hspace{-2.5pt} \rangle}
%%% &=& &=& &=& &=& &=& &=& &=& &=& %%% &=& &=& &=& &=& &=& &=& &=& &=& %%% &=& &=& &=& &=%}}}

\title{......}%{{{
\author{
Zhenzhen Liu$^{1}$\footnote{\emph{E-mail address}: zhzhenliu@gmail.com (Z. Liu).},
Mogei Wang$^{2}$\footnote{Corresponding author. \emph{E-mail address}: mogeiwang@gmail.com (M. Wang).}
\\{\scriptsize{1. School of Information and Communication Engineering, Dalian Nationalities University, 116600, P.R.
China}}
\vspace{-3mm} \\{\scriptsize{2. Independent researcher}} } \date{} \maketitle \vspace{-10mm}

\begin{abstract}
...
\\[5pt]PACS number(s): 89.75.-k, 87.23.Ge.
\\[5pt]{\bf Keywords:} Agent-based modeling, ...
\end{abstract}%}}}

\section{Introduction}%{{{

There is a long history of research on the ... theory~\citep{Boccaletti2006, Liu2010}.

In this work, it is denoted as $\tau$.

%}}}

\section{Model}%{{{

There are $N$ ... which ... $M$ kinds of ...

The position and motion direction of the individual $i$ at time $t$ are denoted
as $\textbf{x}_i(t)$ and $\theta_i(t)$. The individual $i$ moves according to the following equation:
\begin{equation} \label{Eq:...}%{{{
\left\{
\begin{array}{l@{\quad \quad}l}
\textbf{x}_i(t+\Delta t)=\textbf{x}_i(t)+\textbf{v}_i(t)\Delta t \\
\theta_i(t+\Delta t)=\xi_i
\end{array}
\right., \nonumber
\end{equation} %}}}
where $\textbf{v}_i(t)= (v\cos\theta_i(t),v\sin\theta_i(t))$ is the velocity of the individual, and $v$ is the modulus
of the individual motion velocity and is the same for all individuals. $\xi_i$ follows the uniform distribution in
$[-\pi,\pi]$. $\Delta t$ is the update interval and is set to 1. For individuals $i$ and $j$, if
$\left|\textbf{x}_i-\textbf{x}_j\right|\le r_0$ where $r_0$ is the interaction radius, then $i$ and $j$ are neighbors
and can infect each other.

For the individual $i$, if it is currently ...

%}}}

\subsection{...} \label{Sect:...}%{{{
.........

\begin{figure}[htbp]\centering
\subfigure[$\rho_1$ vs $\tau$]{\label{Fig_taua}
\begin{minipage}[h]{0.4\textwidth}
\begin{overpic}[scale=0.7]{....eps}
\end{overpic}
\end{minipage} }
\hspace{0.5cm}
\subfigure[$\rho_2$ vs $\tau$]{\label{Fig_taub}
\begin{minipage}[h]{0.4\textwidth}
\begin{overpic}[scale=0.7]{....eps}
\end{overpic}
\end{minipage} }
\hspace{0.5cm}
\subfigure[$\chi_1$ vs $\tau$]{\label{Fig_tauc}
\begin{minipage}[h]{0.4\textwidth}
\begin{overpic}[scale=0.7]{....eps}
\end{overpic}
\end{minipage} }
\hspace{0.5cm}
\subfigure[$\chi_2$ vs $\tau$]{\label{Fig_taud}
\begin{minipage}[h]{0.4\textwidth}
\begin{overpic}[scale=0.7]{....eps}
\end{overpic}
\end{minipage} }
\vspace{-2mm}
\caption{\label{Fig_rho_chi_tau} ...}
\end{figure}
%}}}

\subsection{...} \label{Sect:...}%{{{
.........
%}}}

\section{...}%{{{
Simulation results and data obtained are analyzed in this section.%}}}

\begin{table}[htop]
\centering
\caption {ratio {...}} \label{tab:ratio}
\begin{tabular}{c|c|c|c}
% after \\: \ hline or \ cline {col1−col2} \cline{col3−col4 } ...
\hline
\backslashbox{\(H/L\)}{\(hd\)} & 0.75 & 1 & 1.25 \\\hline
0.03 & 1.0698 & 1.0035 & 0.9878 \\
0.04 & 2 & 3 & 4 \\
0.05 & 2 & 3 & 4 \\
0.067 & 2 & 3 & 4 \\
0.09 & 2 & 3 & 4 \\
\(\left(\dfrac{H}{L}\right) {max}\) & 2 & 3 & 4 \\\hline
\end{tabular}
\end{table}

\begin{table}[htop]
\centering
\caption {date room} \label{tab:room}
\begin{tabular}{|l|ccc|}
\hline
\diagbox{Time}{Room}{Day} & Mon & Tue & Wed \\
\hline
Morning & used & used &\\
Afternoon & & used & used \\
\hline
\end{tabular}
\end{table}


\section{Conclusions}%{{{
...%}}}

\begin{thebibliography}{99}\scriptsize%{{{
\bibitem{Boccaletti2006}
S. Boccaletti, V. Latora, Y. Moreno, M. Chavez, D.-U. Hwang, Phys. Rep. 424, 175 (2006).

\bibitem{Liu2010}
Z. Liu, X. Wang, M. Wang, Chaos 20, 023128 (2010).
\end{thebibliography}
\end{document}%}}}
