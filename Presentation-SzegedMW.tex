% !TEX TS-program = XeLaTeX
% !TEX encoding = UTF-8 Unicode

%% Copyright 2005-2014 G. W. Knor
%%This work may be distributed and/or modified under the
% conditions of the LaTeX Project Public License, either version 1.3
% of this license or (at your option) any later version.
% The latest version of this license is in
% http://www.latex-project.org/lppl.txt
% and version 1.3 or later is part of all distributions of LaTeX
% version 2005/12/01 or later.
%%This work has the LPPL maintenance status "maintained".
%%The Current Maintainer of this work is G. W. Knor.

% This version has been revised by Mogei Wang for his presentations.
% Ref. https://code.google.com/p/dlutlatextemplate
% 12-15-2015, in Dalian, China

\documentclass[14pt,handout]{beamer}
% use handout or trans to avoid the small buttons at buttom

\usepackage{xltxtra,xunicode}
\usepackage[dvipdfmx]{movie15}
\usepackage{graphicx}
\usepackage{subfigure}
\usepackage{color}
\usepackage{overpic}
\graphicspath{{figures/}}

\makeatletter
    \newcommand\fixeddistanceleft[2][1cm]{{\hb@xt@ #1{#2\hss} }}
    \newcommand\fixeddistancecenter[2][1cm]{{\hb@xt@ #1{\hss#2\hss} }}
    \newcommand\fixeddistanceright[2][1cm]{{\hb@xt@ #1{\hss#2} }}
    \newcommand\fixedunderlineleft[2][1cm]{\underline{\hb@xt@ #1{#2\hss} }}
    \newcommand\fixedunderlinecenter[2][1cm]{\underline{\hb@xt@ #1{\hss#2\hss} }}
    \newcommand\fixedunderlineright[2][1cm]{\underline{\hb@xt@ #1{\hss#2} }}
\makeatother

\setbeamertemplate{theorems}[numbered]
\newtheorem{mytheorem}{Theorem}
\newtheorem{myproposition}[mytheorem]{Proposition}

%%\usetheme[secheader]{Madrid}
%%\usetheme{CambridgeUS}
\usetheme{Szeged}
% Bergen,Boadilla,Madrid,AnnArbor,CambridgeUS,Pittsburgh Rochester.
% Antibes,JuanLesPins,Montpellier,
% Berkeley,PaloAlto, Goettingen,Marburg,Hannover,
% Berlin,Ilmenau,Dresden,Darmstadt, Frankfurt,Singapore,Szeged,
% Copenhagen,Luebeck,Malmoe,Warsaw

\usecolortheme{wolverine}
% default,structure,sidebartab,
% albatross, beetle, crane, dove,fly, seagull, wolverine, beaver

\usecolortheme{rose}
% lily,orchid,rose

\usecolortheme{dolphin}
% whale,seahorse,dolphin

\usefonttheme{professionalfonts}
% structurebold,structuresmallcapsserif,professionalfonts
\setbeamercovered{transparent}

%%%%%%%%%%%%%%%%%%%%%%%%%%%%%%%%%%%%%%%%%%%%%%%%%%%%%%%%%%%%%%%%%%%%%%%%%%%%%%%%%%%%%%%%%%%%%%%%%%%%%
\begin{document}
\title{My research presentation}
\author{Mogei Wang}
\institute[M. Wang]{mogeiwang@gmail.com}
% \date{} %!!!
\date{Jan. 1, 2015}

\begin{frame}
\titlepage
\end{frame}

\section{Introduction}%{{{
\begin{frame}
\begin{figure}[htbp]\centering
\begin{overpic}[scale=0.45]{turbulence.jpg}\end{overpic}
\caption[turbulence]{\label{Fig:turbulence} Turbulence}
\end{figure}
%
\begin{block}{The kernel of complex?} \small
Heisenberg was asked what he would ask God, given the opportunity. His reply was: ``When I meet God, I am going to ask him two questions: Why relativity? And why turbulence? I really believe he will have an answer for the first."\footnote{\small http://en.wikipedia.org/wiki/Turbulence}
\end{block}
\end{frame}

\begin{frame}
\begin{itemize} %\begin{enumerate}
    \item Coupled Map Lattice (Kuznetsov; {{turbulence}})
    \item Game Of Life (Conway; {{Turing Machine}})
    \item Cellular Automata (Wolfram; {{complex patterns}})
    \item Boolean Net (Kauffman; {{life}})
\end{itemize} %\end{enumerate}
\begin{block}{the edge of chaos}%{Is that true?} \small
Some cores keep producing new patterns and spreads them,\\
the spreading processes produce new cores.
\end{block}
\end{frame}

\begin{frame}
\begin{block}{My thinkings}
\begin{itemize} %\begin{enumerate}
    \item What are spreading?
    \item How does the spreading process interplay with network topology?
    \item What is the difference between the cores and the other areas?
\end{itemize} %\end{enumerate}
\end{block}
\end{frame}

\section*{Outline}
\begin{frame}[shrink] %a smaller font
\vspace{1cm}
\tableofcontents
\end{frame}

\section{1: Chaos synchronization} %{{{
\begin{frame} \frametitle{Chaos synchronization}
\begin{itemize} %\begin{enumerate}
    \item Chaos refers to the behavior of dynamical systems that are highly sensitive to initial conditions.
    \item Synchronization is the agreement or correlation of different dynamical processes in time.
    \item Chaos synchronization is the synchronization of chaotic processes.
\end{itemize} %\end{enumerate}
\begin{block}{}
    Unlike the synchronization between periodic systems (frequency-locking, phase-locking), chaos systems can be synchronized in various complex modes.
\end{block}
\end{frame}

\subsection{Symbolic dynamic}
\begin{frame} \frametitle{Symbolic dynamic}
\end{frame}

\subsection{Generalized synchronization}
\begin{frame} \frametitle{Generalized synchronization}
\begin{table}[htop]
\centering
\caption[GSync-Table]{Generalized synchronization} \label{tab:gsync}
\begin{tabular}{c|c|c|c}
\hline
$i$ & $X_i^1$ & $Y_i^\infty$ & $Z_i^\infty$ \\
\hline
0 & 1 &             001001000011111$\ldots$ &             101101111110000$\ldots$ \\
1 & 0 & \underline{11}0100100001111$\ldots$ & \underline{11}0110111111000$\ldots$ \\
2 & 1 & \underline{001}010010000111$\ldots$ & \underline{001}011011111100$\ldots$ \\
3 & 1 & \underline{1101}01001000011$\ldots$ & \underline{1101}01101111110$\ldots$ \\
4 & 1 & \underline{11101}0100100001$\ldots$ & \underline{11101}0110111111$\ldots$ \\
% 5 & 0 & \underline{111101}010010000$\ldots$ & \underline{111101}011011111$\ldots$ \\
% 6 & 0 & \underline{0011101}01001000$\ldots$ & \underline{0011101}01101111$\ldots$ \\
% 7 & 0 & \underline{00011101}0100100$\ldots$ & \underline{00011101}0110111$\ldots$ \\
$\vdots$ & $\vdots$ & $\vdots$ & $\vdots$\\
\hline\end{tabular}\end{table}
\begin{block}{}
The outer information is transfered to and stored in the two response systems.
\end{block}
\end{frame}

\section{2: Spreading}
\begin{frame} \frametitle{Spreading}
When $v>0$, the epidemic also spreads inhomogeneously in both modes, moreover, the epidemic spreading in mode 2 also is more inhomogeneous than in mode 1.
\begin{block}{}
Does the moving speed $v$ has any effects on the inhomogeneity?
\end{block}
\end{frame}

\begin{frame}
\begin{figure}[!htbp]\centering
\subfigure[Mode 1]{\label{Fig_va}
\begin{overpic}[scale=0.45]{mode1-rho-v.eps}
\put(-8,31){$\rho_1$}\put(50,-4){$v$}
\end{overpic}}
\hspace{0.5cm}
\subfigure[Mode 2]{\label{Fig_vb}
\begin{overpic}[scale=0.45]{mode2-rho-v.eps}
\put(-8,31){$\rho_2$}\put(50,-4){$v$}
\end{overpic}}
% \caption{\label{fig:figrhov}Infected densities vs moving speed of individuals $v$}
\end{figure}
\begin{figure}[!htbp]\centering
\subfigure[Mode 1]{\label{Fig_vc}
\begin{overpic}[scale=0.45]{mode1-chi-v.eps}
\put(-8,31){$\chi_1$}\put(50,-4){$v$}
\end{overpic}}
\hspace{0.5cm}
\subfigure[Mode 2]{\label{Fig_vd}
\begin{overpic}[scale=0.45]{mode2-chi-v.eps}
\put(-8,31){$\chi_2$}\put(50,-4){$v$}
\end{overpic}}
% \caption{\label{fig:figchiv}CICSs vs moving speed of individuals $v$}
\end{figure}
\end{frame}

\section{Summary}
\begin{frame}
Thank you for your attention!\\
Questions?
\end{frame}

\begin{frame}
Thank you for your attention!\\
Questions?
\end{frame}

\end{document}
